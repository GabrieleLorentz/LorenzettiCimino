\subsection{Purpose}
If you are looking for an internship or want to offer one, you are in the right place, S\&C will help you!\\
Student\&Companies is a platform that helps match university students looking for internships and companies offering them.\\
The primary objective of the platform are as follow:
\begin{itemize}
    \item Students can search for an internship on their own or can be informed when an internship request compatible with their CV and characteristics is published.
    \item The system will help both student and companies find each other, by a system called "recommendation" that helps find the correspondence between the characteristic offered by the student and the ones asked by the company.
    \item The management of the selection process. 
    \item The monitoring of the execution of the internships, including statistics, feedback, and complaints.
\end{itemize}
\subsubsection{Goals}
\textbf{[G1]:} All unregistered students and companies must be able to subscribe and login to the S\&C platform.\\
\textbf{[G2]:} Students and companies must be able to write their descriptions and preferences.\\
\textbf{[G3]:} Students must be able to complete their CV.\\
\textbf{[G4]:} Companies must be able to create internship offers.\\
\textbf{[G5]:} Students must be able to search for an internship.\\
\textbf{[G6]:} Student and Company are informed when there is a match between them.\\
\textbf{[G7]:} Monitoring of the execution of the internships.\\
\textbf{[G8]:} Statistics collection.\\
\textbf{[G9]:} Feedbacks collection.\\
\textbf{[G10]:} Companies rank based on feedback. 
\subsection{Scope}
S\&C allows student and companies to communicate easily in a guided environment.\\
Students are able to upload their CVs and express their preferences about the work environment.\\
Companies upload their internship offers through the platform.\\
Students need to be able to actively search for an internship, through a keyword or by selecting some preferences.\\
The system  implements a process called "recommendation": it's a mechanisms that apply a research through the internship preferences and students characteristics, to inform students when an interesting internship becomes available and informs also companies that a student matches with theirs preferences.\\
The platform also propose to help companies in the selection process. When there is a match between the two parts, and both of them accept it, the process start. Companies feed questionnaire to the students and collect their responses to evaluate their fit with the company and can finalize the selection. 
The platform also stores statistics about internship that are offered by companies and the feedback from the students.
\subsubsection{Phenomena}
\begin{table}[H]
\renewcommand\arraystretch{1.5}
    \centering
    \begin{tabular}{|ccc|}
        \hline
        \rowcolor{BurntOrange}
        \textbf{Phenomenon}&  \textbf{Who controls it?}& \textbf{Is shared?}\\
        \hline
        Users decides to use S\&C&  W& N\\
        \hline
        User registration&  W& Y\\
        \hline
        User login& W& Y\\
        \hline
        Check username and password&  M& N\\
        \hline
        Student create CV&  W& Y\\
        \hline
        Company create offer&  W& Y\\
        \hline
        Recommendation process start& M& N\\
        \hline
        Student search offer&  W& Y\\
        \hline
        Match notification&  M& Y\\
        \hline
        Match acceptation&  W& Y\\
        \hline
        Student's feedback&  W& Y\\
        \hline
        Statistics collection&  M& N\\
        \hline
        Companies rank computation&  M& N\\
        \hline
        Companies rank publication&  M& Y\\
        \hline
        Interview process start&  M& N\\
        \hline
        Form sending&  M& Y\\
        \hline
        Form compilation&  W& Y\\
        \hline
        Interview schedule&  M& Y\\
        \hline
    \end{tabular}
    \caption{Phenomena}
    \label{Phenomena}
\end{table}
\subsection{Definitions, Acronyms, Abbreviations}
\subsubsection{Definitions}
\subsubsection{Acronyms}
\begin{itemize}
    \item \textbf{S\&C:} Student and Companies
    \item \textbf{CV:} curriculum vitae
    \item \textbf{Zoom:} Zoom platform
    \item \textbf{G:} goal
    \item \textbf{A:} assumption
    \item \textbf{UC:} use cases
    \item \textbf{AD:} activity diagram
\end{itemize}
\subsubsection{Abbreviations}
\subsection{Revision history}
\subsection{Reference Documents}
\begin{itemize}
    \item Slides of the course “Software Engineering 2”.
    \item Michael Jackson, The World and the Machine.
    \item Specification document "01. Assignment RDD AY 2024-2025”
    \item Alloy documentation - https://alloytools.org/documentation.html
\end{itemize}
\subsection{Document Structure}
\begin{enumerate}
    \item \textbf{Introduction:} a description of the problem showing the purpose and the scope of the application. In order to precisely delineate the scope, phenomena and goals related to the problem are identified. In this section information about terms used in this document is also present, along with references and revision history.
    \item \textbf{Overall Descriptions:} a high-level view of the project. The perspective of the product is developed with scenarios and descriptions about requirements of the service interfaces. Product functions describe the required functions of the system in order to fulfill the goals as specified by the stakeholders. Furthermore, possible actors are also identified in the user user characteristics section. In the end there is a list of the taken domain assumptions.
    \item \textbf{Specific Requirements:} detailed description of the user interface, function and non-functional service requirements specification. Functional Requirements are supported from specific use cases and mappings that permit to acknowledge how the goals are satisfied.
    \item \textbf{Formal Analysis Using Alloy:} Alloy model useful in checking necessary properties of the system, and generating possible world in which the same will operate.
    \item \textbf{Effort Spent:} hours spent by each group member on the various activities related to the document developing.
\end{enumerate}
%what you write here is a comment that is not shown in the final text